%&pdfLaTeX
% !TEX encoding = UTF-8 Unicode
\documentclass{article}
\usepackage{listings}
\usepackage{marginnote}
\usepackage[top=1.5cm, bottom=1.5cm, outer=5cm, inner=2cm, heightrounded, marginparwidth=2.5cm, marginparsep=2cm]{geometry}
\usepackage{ifxetex}
\ifxetex
\usepackage{fontspec}

\setmainfont[Mapping=tex-text]{STIXGeneral}
\else
\usepackage[T1]{fontenc}
\usepackage[utf8]{inputenc}
\fi
\usepackage{textcomp}

\usepackage{ulem}
\usepackage{amssymb}
\usepackage{fancyhdr}
\renewcommand{\headrulewidth}{0pt}
\renewcommand{\footrulewidth}{0pt}
\usepackage{color}

\definecolor{color02}{rgb}{0.03,0.23,0.42}
\definecolor{color04}{rgb}{1.00,1.00,1.00}
\definecolor{color05}{rgb}{0.30,0.30,0.30}
\definecolor{color06}{rgb}{0.64,0.00,0.01}
\definecolor{color07}{rgb}{0.33,0.84,0.16}
\definecolor{color08}{rgb}{0.07,0.55,1.00}
\definecolor{color09}{rgb}{0.50,0.00,1.00}

\begin{document}

\baselineskip=14pt
{\large{}{\color{color02} Introduction}}

The objective of this assignment was to learn how stack operations are done and
to give us more practice with MIPS assembly language by learning to implement nested
routines. In C, this is equivalent to nested for loops. The challenge for this
lab, however, was to use stack pointers instead of storing information directly
in registers. This lab was an introduction to push, pop, and register addressing,
as well as more practice with memory addressing.

{\large{}{\color{color02} Solution Methodology}}

For this assignment, we wrote a main routine in which two 6\ensuremath{\times}1
vectors were defined in a similar manner to how numbers\_to\_use: was implemented
in lab 0, with some of the values being negative numbers. We configured the stack
pointer and passed the parameters to a function called dot\_product which returned
the result to main, as well as the two vector average values. All of this was done
with overflow checking in mind. (see \emph{Source Code}). With the routine built,
we attached and programmed the chipKit Pro MX4 board and ran the instructions.
Initially, the program performed the mathematical functions and went into an infinite
loop (see \emph{Figure 1}). Below is an example of the pseudocode used to design
this recursive operation.

\begin{center}
\baselineskip=1pt
{\color{color04} Pseudocode}
\end{center}

\baselineskip=14pt
\leftskip=120pt
int dot (int i)

\{

\parindent=18pt
if (i \texttt{<} 6)

\parindent=36pt
vector1(i)*vector2(i);

\parindent=18pt
else sum(vector1(i)+vector2(i));

\parindent=0pt
\}

\vspace{14pt}
\leftskip=0pt
One of the reasons for using the stack is to avoid delay hazards which results
in registers being called before they are finished being executed. One solution
is to push all registers that must be preserved onto the stack, just as we did
with the saved registers in lab 0. The caller pushes any argument registers (\$a0--\$a3)
or temporary registers (\$t0--\$t9) that are needed after the call. The callee
pushes the return address register \$ra and any saved registers (\$s0--\$s7) used
by the callee. The stack pointer \$sp is adjusted to account for number of registers
placed on the stack. Upon the return, the registers are restored from memory and
the stack pointer is readjusted.

\begin{center}
\textbf{Figure 1}
\end{center}

\baselineskip=14pt
\leftskip=0pt
These values are consistent with the calculations as performed in MATLAB (\emph{Figure
2}).

\begin{center}
\textbf{Figure 2}
\end{center}

\baselineskip=14pt
\leftskip=0pt
{\large{}{\color{color02} Source Code}}

\lstdefinestyle{customc}{
  belowcaptionskip=1\baselineskip,
  breaklines=true,
  frame=L,
  xleftmargin=\parindent,
  language=C,
  showstringspaces=false,
  basicstyle=\footnotesize\ttfamily,
  keywordstyle=\bfseries\color{green!40!black},
  commentstyle=\itshape\color{purple!40!black},
  identifierstyle=\color{blue},
  stringstyle=\color{orange},
}

\lstdefinestyle{customasm}{
  belowcaptionskip=1\baselineskip,
  frame=L,
  xleftmargin=\parindent,
  language=[x86masm]Assembler,
  basicstyle=\footnotesize\ttfamily,
  commentstyle=\itshape\color{purple!40!black},
}

\lstset{escapechar=@,style=customc}

\lstset{
  language=C,                     % choose the language of the code
  numbers=left,                   % where to put the line-numbers
  stepnumber=2,                   % the step between two line-numbers.
  numbersep=5pt,                  % how far the line-numbers are from the code
  backgroundcolor=\color{white},  % choose the background color. You must add \usepackage{color}
  showspaces=false,               % show spaces adding particular underscores
  showstringspaces=false,         % underline spaces within strings
  showtabs=false,                 % show tabs within strings adding particular underscores
  tabsize=2,                      % sets default tabsize to 2 spaces
  captionpos=b,                   % sets the caption-position to bottom
  breaklines=true,                % sets automatic line breaking
  breakatwhitespace=true,         % sets if automatic breaks should only happen at whitespace
  title=\lstname,                 % show the filename of files included with \lstinputlisting;
}

\lstinputlisting{kirby_lab03.c}

{\large{}{\color{color02} Conclusion}}

Laboratory 2 was designed to familiarize students with stack operations and to
give us more practice with MIPS assembly language by learning to implement nested
routines. This was critical to understanding how to properly use push and pop to
store and retrieve data as necessary. We also again made use of various registers
and the HI and LO portions of the multiplication function to detect overflows.

\newpage

\end{document}
