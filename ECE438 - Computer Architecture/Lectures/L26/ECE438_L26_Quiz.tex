\documentclass{article}
\usepackage[utf8]{inputenc}
\usepackage[T1]{fontenc} % uses T1 fonts (better quality)
\usepackage{lmodern} % uses Latin Modern fonts
\usepackage[dvipsnames]{xcolor}
\usepackage[margin=1in]{geometry}
\usepackage{nopageno} % no page numbers
\usepackage{graphicx}
\usepackage{minted} % code syntax
\usemintedstyle{vs}

\title{}
\begin{document}
% \begin{titlepage}
% \vspace*{\fill}
 	\begin{center}
     \line(1,0){300}\\[0.25cm]
 	\LARGE{\bfseries Lecture 26 Quiz}\\
 	\textsc{\Large David Kirby}\\
 	\textsc{\large Due: 08 May 2020}\\
 	\line(1,0){300}\\[0.75cm]
 	\end{center}
 %	\vfill
% \end{titlepage}

\begin{enumerate}
\bfseries \item Define least recently used block strategy.\par
\mdseries We have a choice of which entry in a set to replace on a miss. Typically, the block that was used furthest in the past is replaced. This is the \textit{least recently used} strategy.

\textbf{\item What are the advantages of a split-cache system?}\par
Split caches, as opposed to unified caches, separate instructions and data, such as with Harvard architectures. An advantage of this is  customization of the level cache for the given data stream (i.e. if certain techniques work really well for an instruction cache vs. data cache, we can finely tune the cache to suit that need).

\textbf{\item Describe the benefits of virtual memory.}\par
\begin{itemize}
    \item Copes with small amounts of main memory, by swapping data between disk and main memory automatically,
    \item allows multiple processes to share resources easily,
    \item enables each process or program to have its own virtual view of memory, and
    \item provides protection between various processes and between processes and the OS.
\end{itemize}
\end{enumerate}
\end{document}