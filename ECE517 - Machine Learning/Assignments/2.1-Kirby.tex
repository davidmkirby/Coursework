\documentclass[11pt]{article}
\usepackage[utf8]{inputenc}
\usepackage[T1]{fontenc} % uses T1 fonts (better quality)
\usepackage{lmodern} % uses Latin Modern fonts
\usepackage[margin=1in]{geometry}
\usepackage[dvipsnames]{xcolor}
\usepackage{ragged2e}
\renewcommand{\baselinestretch}{1.15}
\usepackage{tikz}
\usetikzlibrary{automata,scopes,shapes,matrix,arrows,decorations.pathmorphing}
\tikzset{>={stealth}}
\usepackage{amsmath,amsfonts,amssymb}
\usepackage{bm}
\usepackage{graphicx}
\usepackage[makeroom]{cancel}
\definecolor{OuterBlue}{HTML}{1370AA}
\definecolor{InnerBlue}{HTML}{9BC4DD}
\usepackage{pdfpages}

\begin{document}
    \begin{center}
    \line(1,0){300}\\[0.25cm]
 	\Large{\bfseries ECE517: Assignment 2.1}\\
 	\textsc{\large David Kirby}\\
 	\textsc{\large Due: 09 September 2020}\\
 	\line(1,0){300}\\[0.75cm]
     \end{center}

\noindent A variation of the MMSE criterion minimizes the norm of the weight vector \textbf{w}. This is a way to control the complexity of the structure. The corresponding function is
\begin{align}
    \mathcal{L}\left(\bm{\mathrm {x,w}}\right)=\mathbb{E}\left[e^2 \right]+\bm{\lambda} \parallel \bm{\mathrm {w}}\parallel ^2
\end{align}
\begin{enumerate}
    \item Make the derivation of the closed solution for \textbf{w}.
    \item Work out an iterative solution using the same technique as used in the Least Mean Squares algorithm.
    \item Comment and compare both solutions in a short conclusion section.
\end{enumerate}
The derivations must be complete and the solution should be briefly but completely explained. See the rubric for this and any other homework.
\end{document}