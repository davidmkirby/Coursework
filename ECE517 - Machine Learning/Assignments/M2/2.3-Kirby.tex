\documentclass[11pt]{article}
\usepackage[utf8]{inputenc}
\usepackage[T1]{fontenc} % uses T1 fonts (better quality)
\usepackage{lmodern} % uses Latin Modern fonts
\usepackage[margin=1in]{geometry}
\usepackage[dvipsnames]{xcolor}
\usepackage{ragged2e}
\renewcommand{\baselinestretch}{1.15}
\usepackage{tikz}
\usetikzlibrary{automata,scopes,shapes,matrix,arrows,decorations.pathmorphing}
\tikzset{>={stealth}}
\usepackage{amsmath,amsfonts,amssymb}
\usepackage{bm}
\usepackage{graphicx}
\usepackage[makeroom]{cancel}
\definecolor{OuterBlue}{HTML}{1370AA}
\definecolor{InnerBlue}{HTML}{9BC4DD}
\usepackage{pdfpages}

\begin{document}
    \begin{center}
    \line(1,0){300}\\[0.25cm]
 	\Large{\bfseries ECE517: Assignment 2.2}\\
 	\textsc{\large David Kirby}\\
 	\textsc{\large Due: 09 September 2020}\\
 	\line(1,0){300}\\[0.75cm]
     \end{center}

\noindent Write a complete derivation of the proof that the maximization of margin \textit{d} is equivalent to minimize the norm of the parameter vector in the maximum margin machine presented in the last set of slides of this module.\par

\noindent The equation for the hyperplane bisecting our data is
\begin{align}
	\bm{\mathrm{w^\top x_{0}}}+b=0
\end{align}
where \textbf{w} is the norm of the parameter vector. From there, we are able to determine two support vectors on either side corresponding to our data +1 and \(-\)1, as well as the margin. These equations are respectively
\begin{align}
	\bm{\mathrm{w^\top x_{0}}}+b&=+1\\
	\bm{\mathrm{w^\top x_{0}}}+b&=-1
\end{align}
The distance from these support vectors to the hyperplane is the margin and given by
\begin{align}
	\bm{\mathrm{x_1}} &=\bm{\mathrm{x_{0}}}+\rho\bm{\mathrm{w}}\\
	d&=\parallel\bm{\mathrm{x_1}}-\bm{\mathrm{x_{0}}}\parallel=\rho\bm{\mathrm{w}}
\end{align}
where \(\bm{\mathrm{x_{0}}}\) is a point on the hyperplane and \(\bm{\mathrm{x_1}}\) is a point normal to \(\bm{\mathrm{x_{0}}}\) and \(\rho=\frac{1}{\parallel\bm{\mathrm{w}}\parallel^2}\). Using equations (1) and (5), we can solve for \textit{d} in terms of the norm
\begin{align}
	d&=\rho\parallel\bm{\mathrm{w}}\parallel=\frac{1}{\parallel\bm{\mathrm{w}}\parallel}
\end{align}
From equation (6) we can see that as \(\parallel\bm{\mathrm{w}}\parallel \) goes to infinity, \textit{d} goes to 0 and vice versa, thus proving that the maximization of margin \textit{d} is equivalent to the minimization of the norm of the parameter vector.
\end{document}