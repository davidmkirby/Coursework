% !TEX program = xelatex
\documentclass[11pt]{article}
\usepackage[margin=1in]{geometry}
\usepackage{nopageno} % no page numbers
\usepackage{setspace} % \doublespacing

\usepackage{graphicx}
\graphicspath{ {./graphics/} }
\usepackage[dvipsnames]{xcolor}
\definecolor{CrispBlue}{HTML}{0176AE}

\usepackage{fontspec}
\usepackage{tcolorbox}
\usepackage{etoolbox}
\BeforeBeginEnvironment{verbatim*}{\begin{tcolorbox}[colback=CrispBlue!5!white,colframe=CrispBlue!75!black]}%
\AfterEndEnvironment{verbatim*}{\end{tcolorbox}}%

\usepackage{hyperref}
\hypersetup{
    colorlinks,
    citecolor=black,
    filecolor=black,
    linkcolor=black,
    urlcolor=black
}

\usepackage[makeroom]{cancel}

\renewcommand{\footnotesize}{\fontsize{8pt}{10pt}\selectfont}

% \usepackage[labelfont={small,sc,bf},textfont={small,sc,bf}]{caption}
\setlength{\parindent}{0pt}
% \setlength{\parskip}{1em}

\usepackage{tocloft}
\renewcommand{\cftpartleader}{\cftdotfill{\cftdotsep}}
\renewcommand{\cftsecleader}{\cftdotfill{\cftdotsep}}

\usepackage[shortlabels]{enumitem}

\usepackage{lastpage}
\usepackage{fancyhdr}
\pagestyle{fancy}
\fancyhf{}
\renewcommand{\headrulewidth}{0pt}
\rfoot{Page \thepage\ of \pageref*{LastPage}}

\usepackage{amsmath,amsfonts,amssymb}
\usepackage{bm}
\usepackage{mathtools}

\renewcommand{\listfigurename}{List of Figures}

\begin{document}
% \setmainfont{SF Pro Text}
% \setsansfont{SF Pro Text}
% \setmonofont{SF Mono}
% \renewcommand{\familydefault}{\sfdefault}

\hypersetup{
    linkcolor=CrispBlue,
    urlcolor=CrispBlue,
    breaklinks=true
}

David Kirby\\
ECE 520: VLSI Design\\
Spring 2022

\begin{center}
    \large\bfseries Homework \#2
\end{center}

\begin{enumerate}
    \item An NMOS transistor has a threshold \(V_T\) of 0.5 V when its source--to--substrate voltage is zero, given that the substrate is uniformly doped at 2E17 acceptor dopant atm/cm\(^3\) and the gate oxide capacitance is 3.5 fF/\(\mu\)m\(^2\).
    \begin{enumerate}
        \item Determine an expression for the threshold voltage as a function of source--to--substrate voltage.\vspace{1.5em}
        
        Using the equation for body effect:
        \begin{align}
            V_T &= V_{T_0} + \gamma \left(\sqrt{|2 \phi_F - V_{BS}|}-\sqrt{|2 \phi_F|}\right)\\[1em]
            \text{where } \gamma &= \frac{\sqrt{2qN_A\epsilon_{Si}}}{C_{ox}}
        \end{align}
        \item It is desired to obtain a threshold voltage of 1.0 volt at 0 volts source potential (with respect to ground). One method suggested by engineering team is to provide a separate bias supply for the substrate, in order to increase the source-to-substrate voltage. What value of Vx supply is needed?\vspace{1.5em}
        
        Using equation (1) from part (a), plugging in values, we get:
        \begin{align*}
            V_T &= V_{T_0} + \gamma \left(\sqrt{|2 \phi_F - V_{BS}|}-\sqrt{|2 \phi_F|}\right)\\[1em]
            1 \text{V} &= 0.5 \text{V} + 0.736 \left(\sqrt{|0.87 - V_{BS}|}-\sqrt{|0.87|}\right)\\[1em]
            V_{BS} &= -1.73 \text{V} \text{ or } \cancel{3.47 \text{V}}\\[1em]
            \text{therefore } V_x &= V_B = -1.73 \text{V}
        \end{align*}
        \item Rather than use a separate substrate bias generator, another group in engineering is suggesting to use a threshold adjustment implant in the fabrication. Assuming the implant acts as a sheet charge in the oxide-silicon interface (via the term Q\(_{\text{fc}}\)), what dose is needed to obtain V\(_{TN}\) = 1 volt at V\(_{SB}\) = 0? Would you use acceptor (N\(_A\)) or donor (N\(_D\)) atoms?
        \begin{align}
            V_T &= \frac{Q}{C_{ox}} - V_{T_0}\\[1em]
            \notag 0.5 \text{V} &= \frac{Q'}{3.5 \text{ fF}/\mu \text{m}^2} - 0\\[1em]
            \notag Q' &= 1.75 \text{ V fF}/\mu \text{m}^2
        \end{align}
        This equates to \(1.09 \times 10^{12}\) cm\(^{-3}\) acceptor (N\(_A\)) atoms.
    \end{enumerate}
\end{enumerate}
\end{document}