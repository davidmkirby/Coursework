% !TEX program = xelatex
\documentclass[11pt]{article}
\usepackage[margin=1in]{geometry}
\usepackage{nopageno} % no page numbers
\usepackage{setspace}

\usepackage{graphicx}
\graphicspath{ {./graphics/} }
\usepackage[dvipsnames]{xcolor}
\definecolor{CrispBlue}{HTML}{0176AE}

\usepackage{fontspec}
\usepackage{tcolorbox}
\usepackage{etoolbox}
\BeforeBeginEnvironment{verbatim*}{\begin{tcolorbox}[colback=CrispBlue!5!white,colframe=CrispBlue!75!black]}%
\AfterEndEnvironment{verbatim*}{\end{tcolorbox}}%

\usepackage{hyperref}
\hypersetup{
    colorlinks,
    citecolor=black,
    filecolor=black,
    linkcolor=black,
    urlcolor=black
}

\usepackage{subcaption}
\setlength{\parindent}{0pt}
\setlength{\parskip}{1em}

\usepackage{tocloft}
\renewcommand{\cftpartleader}{\cftdotfill{\cftdotsep}}
\renewcommand{\cftsecleader}{\cftdotfill{\cftdotsep}}

\usepackage[shortlabels]{enumitem}

\usepackage{fancyhdr}
\pagestyle{fancy}
\fancyhf{}
\lhead{ECE 529: Cybersecurity}
\rhead{Assignment 2}
\rfoot{Page \thepage}

\usepackage{amsmath,amsfonts,amssymb}
\usepackage{bm}
\usepackage{mathtools}

\renewcommand{\listfigurename}{List of Figures}

\begin{document}
\setmainfont{SF Pro Text}
\setsansfont{SF Pro Text}
\setmonofont{SF Mono}
\renewcommand{\familydefault}{\sfdefault}

\thispagestyle{empty}
\begin{titlepage}
\vspace*{\fill}
\begin{center}
\textsc{\Huge{ECE 529: Cybersecurity}}\\[3em]
\textsc{\LARGE Assignment 2: Vulnerability Exploitation}\\[6em]
\textsc{\Large David Kirby -- 101652098 -- davidkirby@unm.edu}\\[3em]
\textsc{\Large Spring 2022}
\end{center}
\vfill
\begin{figure}[h]
\begin{subfigure}{0.5\textwidth}
\includegraphics[width=0.20\linewidth]{cybersecurity.png}
\end{subfigure}
\begin{subfigure}{0.6\textwidth}\hspace{1em}
\includegraphics[width=0.8\linewidth]{new-soe-logo.png}
\end{subfigure}
\end{figure}
\end{titlepage}
\setcounter{figure}{0}

\hypersetup{
    linkcolor=CrispBlue,
    urlcolor=CrispBlue,
    breaklinks=true
}

\textbf{GAUSSIAN PROCESSES}

Using the GP software and the script provided with video 7.4, reproduce the second example of video 7.3. The code to produce the data is the following. To produce different data, change the seed number.

[Code omitted for brevity.]


% \begin{tcolorbox}[colback=CrispBlue!5!white,colframe=CrispBlue!75!black,title=Gaussian Processes]\setstretch{1.25}


% \end{tcolorbox}

\begin{figure}[h!]
    \centering
    \includegraphics[width=\textwidth]{figure01.png}
    \caption{Linear prediction from past samples with \(\sigma = 0.05\).}
    \label{fig:fig01}
\end{figure}

\begin{figure}[h!]
    \centering
    \includegraphics[width=\textwidth]{figure02.png}
    \caption{Linear prediction from past samples with \(\sigma = 0.01\).}
    \label{fig:fig02}
\end{figure}

\end{document}