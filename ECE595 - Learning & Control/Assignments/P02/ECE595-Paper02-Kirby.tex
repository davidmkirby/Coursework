\documentclass[11pt]{article}
\usepackage[utf8]{inputenc}
\usepackage[T1]{fontenc} % uses T1 fonts (better quality)
\usepackage{lmodern} % uses Latin Modern fonts
\usepackage[margin=1in]{geometry}
\usepackage[dvipsnames]{xcolor}
\usepackage{ragged2e}
\renewcommand{\baselinestretch}{1.15}
\usepackage{tikz}
\usetikzlibrary{automata,scopes,shapes,matrix,arrows,decorations.pathmorphing}
\tikzset{>={stealth}}
\usepackage{mathtools}
\usepackage{bm}
\usepackage{graphicx}
\usepackage[makeroom]{cancel}
\usepackage{pdfpages}
\usepackage{amssymb}
\usepackage{gensymb}
\usepackage{rotating}
\definecolor{CrispBlue}{HTML}{0176AE}


\begin{document}
\begin{center}
\LARGE{ECE 595: Learning and Control\\Paper \#2 Summary}\\[1.5em]
\large David Kirby\\[1.5em]
\large \textbf{Due Friday, February 12, 2021 at 9:00 AM}\\[2.5em]
\end{center}

\noindent This paper attempts to address uncertainty that can arise in model predictive control problems (i.e. errors that occur in learning due to lack of data or errors inherent to the system). The authors claim that their use of quantile regression, enforcing monotonicity, and epistemic uncertainty create a novel approach to handling uncertainty with respect to MPCs by creating deep learning ``tubes'' derived directly from data. The authors present previous related works that offer other solutions, but claim that this is the first to use deep neural networks with MPCs. As discussed in class, this method seems to be limited in scope and robustness as it appears to only apply to well-behaved and well-defined control systems.

\end{document}