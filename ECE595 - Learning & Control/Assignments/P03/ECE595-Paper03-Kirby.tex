\documentclass[11pt]{article}
\usepackage[utf8]{inputenc}
\usepackage[T1]{fontenc} % uses T1 fonts (better quality)
\usepackage{lmodern} % uses Latin Modern fonts
\usepackage[margin=1in]{geometry}
\usepackage[dvipsnames]{xcolor}
\usepackage{ragged2e}
\renewcommand{\baselinestretch}{1.15}
\usepackage{tikz}
\usetikzlibrary{automata,scopes,shapes,matrix,arrows,decorations.pathmorphing}
\tikzset{>={stealth}}
\usepackage{mathtools}
\usepackage{bm}
\usepackage{graphicx}
\usepackage[makeroom]{cancel}
\usepackage{pdfpages}
\usepackage{amssymb}
\usepackage{gensymb}
\usepackage{rotating}
\definecolor{CrispBlue}{HTML}{0176AE}


\begin{document}
\begin{center}
\LARGE{ECE 595: Learning and Control\\Paper \#3 Summary}\\[1.5em]
\large David Kirby\\[1.5em]
\large \textbf{Due Friday, February 19, 2021 at 9:00 AM}\\[2.5em]
\end{center}

\noindent Similar to paper \#2, this paper presents a novel approach to dealing with uncertainty that comes with control systems. These can come from state estimation, disturbances, modeling errors, and stochastic issues such as component failure. The potential use cases as presented by the authors include unmanned aircraft subject to turbulence and wheeled robotic vehicles prone to component failures. This immediately made me think of the Mars Perseverance landing on Thursday. It makes me wonder if the helicopter and rover implement some of this research since the paper's first author is with JPL. The authors do a thorough job of describing the relevant work and describing the novelty of their work. They present recent literature that attempt to set boundaries on uncertainty, but the authors argue that many disturbances such as wind are better modeled under stochastic modeling than with set bounds. One particular aspect that I was interested in was the paper's discussion of Markov decision processes, where ``for discrete-state spaces, value iteration can be used to find the control policy that maximizes the expected reward.'' This reminded me of Dr. Tsiropoulou's class in which we studied game theory and its applications to network dynamics. Are these similar? The authors' most important presentation, though, is their use of joint chance constraints to limit the probability of failure beyond a certain threshold. Their use of joint distributions allows for more robustness in terms of what noise the system can model. Their novelty lies in the use of jump Markov linear systems and mixed-integer linear programming, allowing for continuous chance-constrained uncertainty variables.

\end{document}