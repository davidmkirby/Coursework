\documentclass[11pt]{article}
\usepackage[utf8]{inputenc}
\usepackage[T1]{fontenc} % uses T1 fonts (better quality)
\usepackage{lmodern} % uses Latin Modern fonts
\usepackage[margin=1in]{geometry}
\usepackage[dvipsnames]{xcolor}
\usepackage{ragged2e}
\renewcommand{\baselinestretch}{1.15}
\usepackage{tikz}
\usetikzlibrary{automata,scopes,shapes,matrix,arrows,decorations.pathmorphing}
\tikzset{>={stealth}}
\usepackage{mathtools}
\usepackage{bm}
\usepackage{graphicx}
\usepackage[makeroom]{cancel}
\usepackage{pdfpages}
\usepackage{amssymb}
\usepackage{gensymb}
\usepackage{rotating}
\usepackage{hyperref}
\definecolor{CrispBlue}{HTML}{0176AE}
\hypersetup{
    colorlinks=true,
    linkcolor=CrispBlue,
    urlcolor=CrispBlue,
    breaklinks=true
}

\begin{document}
\begin{center}
\LARGE{ECE 595: Learning and Control\\Paper \#7 Summary}\\[1.5em]
\large David Kirby\\[1.5em]
\large \textbf{Due Saturday, April 24, 2021 at 9:00 AM}\\[2.5em]
\end{center}

\noindent The authors of this paper have developed a way to calculate ambiguity sets of unknown probability distributions using only data collected from dynamically varying processes. They determine growth rate and sampling rate to ensure the ambiguity sets converge. They have also accounted for disturbances in the dynamics by exploiting past samples. The authors then generalize this partial-state measurements of linear time-varying systems.
\end{document}