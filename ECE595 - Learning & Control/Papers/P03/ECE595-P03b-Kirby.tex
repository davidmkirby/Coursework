\documentclass[11pt]{article}
\usepackage[utf8]{inputenc}
\usepackage[T1]{fontenc} % uses T1 fonts (better quality)
\usepackage{lmodern} % uses Latin Modern fonts
\usepackage[margin=1in]{geometry}
\usepackage[dvipsnames]{xcolor}
\usepackage{ragged2e}
\renewcommand{\baselinestretch}{1.15}
\usepackage{tikz}
\usetikzlibrary{automata,scopes,shapes,matrix,arrows,decorations.pathmorphing}
\tikzset{>={stealth}}
\usepackage{mathtools}
\usepackage{bm}
\usepackage{graphicx}
\usepackage[makeroom]{cancel}
\usepackage{pdfpages}
\usepackage{amssymb}
\usepackage{gensymb}
\usepackage{rotating}
\usepackage{hyperref}
\definecolor{CrispBlue}{HTML}{0176AE}
\hypersetup{
    colorlinks=true,
    linkcolor=CrispBlue,
    urlcolor=CrispBlue,
    breaklinks=true
}

\begin{document}
\begin{center}
\LARGE{ECE 595: Learning and Control\\Paper \#3b Summary}\\[1.5em]
\large David Kirby\\[1.5em]
\large \textbf{Due Friday, February 26, 2021 at 9:00 AM}\\[2.5em]
\end{center}

\noindent This week's paper is by the same first author as last week's. It's no surprise then that the topic is similar -- using chance-constraints to optimize paths. This paper however does not use ``particles'' to predict the path, but instead implements a disjunctive convex program to approximate bounds. I'm not familiar with disjunctive convex problems, but the paper gave a sufficient overview. According to the authors, the particle method from last week was appropriate for arbitrary uncertainty distributions; however, this paper points out that the particle method can be computationally intensive. The method proposed in this week's paper, while primarily useful for Gaussian distributions, is much more computationally efficient while also solving an issue the particle method had of being too conservative. It relaxes conservativism by using fixed values for the chance constraints, one for risk tightening (creating an upper bound) and another for risk relaxation (creating a lower bound).

This paper reminded me of my undergraduate work at CNM where in 2017 I participated in the \href{http://nasaswarmathon.com}{Swarmathon robotics competition}. The competition is run by Dr. Melanie Moses at UNM and is sponsored by NASA. In it, we were given autonomous robots (swarmies) for which we were tasked to program a search algorithm. The robots would then be let loose into a defined area to search for QR-coded cubes. The swarmies would need to pick up the QR cubes and return them to a home base. The more difficult stages of the competition (held at NASA's Kennedy Space Center) incorporated obstacles that the swarmies would have to avoid. This paper could have been incredibly helpful.

\end{document}