\documentclass[11pt]{article}
\usepackage[utf8]{inputenc}
\usepackage[T1]{fontenc} % uses T1 fonts (better quality)
\usepackage{lmodern} % uses Latin Modern fonts
\usepackage[margin=1in]{geometry}
\usepackage[dvipsnames]{xcolor}
\usepackage{ragged2e}
\renewcommand{\baselinestretch}{1.15}
\usepackage{tikz}
\usetikzlibrary{automata,scopes,shapes,matrix,arrows,decorations.pathmorphing}
\tikzset{>={stealth}}
\usepackage{mathtools}
\usepackage{bm}
\usepackage{graphicx}
\usepackage[makeroom]{cancel}
\usepackage{pdfpages}
\usepackage{amssymb}
\usepackage{gensymb}
\usepackage{rotating}
\usepackage{hyperref}
\definecolor{CrispBlue}{HTML}{0176AE}
\hypersetup{
    colorlinks=true,
    linkcolor=CrispBlue,
    urlcolor=CrispBlue,
    breaklinks=true
}

\begin{document}
\begin{center}
\LARGE{ECE 595: Learning and Control\\Paper \#4 Summary}\\[1.5em]
\large David Kirby\\[1.5em]
\large \textbf{Due Friday, March 12, 2021 at 9:00 AM}\\[2.5em]
\end{center}

\noindent To me this paper seemed more like what we would find in a traditional textbook. It gives a general overview of Gaussian Processes and their practical uses for controls including generative models, Bayesian inference, and regression. The paper then goes into further depth for various use cases, including tutorials. Gaussian Processes are described as ideal choices for approximating nonlinear functions due to their expressiveness and flexibility; however, the authors describe the limitations of Gaussian Processes, namely that outputs must be scalar or breaks down when the input data is dynamic and nonstationary (multiple switching processes).

\end{document}