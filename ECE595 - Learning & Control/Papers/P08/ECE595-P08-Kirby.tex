\documentclass[11pt]{article}
\usepackage[utf8]{inputenc}
\usepackage[T1]{fontenc} % uses T1 fonts (better quality)
\usepackage{lmodern} % uses Latin Modern fonts
\usepackage[margin=1in]{geometry}
\usepackage[dvipsnames]{xcolor}
\usepackage{ragged2e}
\renewcommand{\baselinestretch}{1.15}
\usepackage{tikz}
\usetikzlibrary{automata,scopes,shapes,matrix,arrows,decorations.pathmorphing}
\tikzset{>={stealth}}
\usepackage{mathtools}
\usepackage{bm}
\usepackage{graphicx}
\usepackage[makeroom]{cancel}
\usepackage{pdfpages}
\usepackage{amssymb}
\usepackage{gensymb}
\usepackage{rotating}
\usepackage{hyperref}
\definecolor{CrispBlue}{HTML}{0176AE}
\hypersetup{
    colorlinks=true,
    linkcolor=CrispBlue,
    urlcolor=CrispBlue,
    breaklinks=true
}

\begin{document}
\begin{center}
\LARGE{ECE 595: Learning and Control\\Paper \#8 Summary}\\[1.5em]
\large David Kirby\\[1.5em]
\large \textbf{Due Friday, April 30, 2021 at 9:00 AM}\\[2.5em]
\end{center}

\noindent The authors of this paper make use of previous work in certificate functions to develop algorithms which show that certificates can actually be derived from learned data and that these certificates can then used to guarantee stability. They ``establish bounds on the generalization error -- the probability that a certificate will not certify a new, unseen trajectory'' and generalize these bounds into global stability guarantees for use in adaptive controls. The certificates implement Lyapunov stability analysis and contraction methods. The paper cites many major previous works and rely heavily on them, including works on learning barrier functions from data, and contracting vector fields. This paper is also a truncated version of a 30-page paper, shortened for publication. The paper claims standing stability for robotics as a practical use case.

\end{document}