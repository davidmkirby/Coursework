\documentclass[11pt]{article}
\usepackage[utf8]{inputenc}
\usepackage[T1]{fontenc} % uses T1 fonts (better quality)
\usepackage{lmodern} % uses Latin Modern fonts
\usepackage[margin=1in]{geometry}
\usepackage[dvipsnames]{xcolor}
\usepackage{ragged2e}
\renewcommand{\baselinestretch}{1.15}
\usepackage{tikz}
\usetikzlibrary{automata,scopes,shapes,matrix,arrows,decorations.pathmorphing}
\tikzset{>={stealth}}
\usepackage{mathtools}
\usepackage{bm}
\usepackage{graphicx}
\usepackage[makeroom]{cancel}
\usepackage{pdfpages}
\usepackage{amssymb}
\usepackage{gensymb}
\usepackage{rotating}
\usepackage{hyperref}
\definecolor{CrispBlue}{HTML}{0176AE}
\hypersetup{
    colorlinks=true,
    linkcolor=CrispBlue,
    urlcolor=CrispBlue,
    breaklinks=true
}

\begin{document}
\begin{center}
\LARGE{ECE 595: Learning and Control\\Paper \#9 Summary}\\[1.5em]
\large David Kirby\\[1.5em]
\large \textbf{Due Friday, May 7, 2021 at 9:00 AM}\\[2.5em]
\end{center}

\noindent The authors of this paper have developed a way to use model-based shared control (MbSC) without relying on a priori knowledge of the system dynamics. They learn the system dynamics and information about the user interaction with the system directly from data. This is done by learning the model through an approximation to the Koopman operator, an infinite dimensional linear operator that can exactly model non-linear dynamics. These learned system models are able to be used in shared control systems that generalize fairly well.
\end{document}