\documentclass[conference]{IEEEtran}

% *** GRAPHICS RELATED PACKAGES ***
\ifCLASSINFOpdf
\else
\fi

% correct bad hyphenation here
\hyphenation{op-tical net-works semi-conduc-tor}

\begin{document}

\title{Internet Economics and Network Neutrality:\\
 A Historical Survey}

% author names and affiliations
\author{\IEEEauthorblockN{Pablo Carabana-Garcia}
\IEEEauthorblockA{School of Electrical and\\Computer Engineering\\
University of New Mexico\\
Albuquerque, New Mexico 87131-0001\\
pablocarabana@unm.edu}
\and
\IEEEauthorblockN{Benjamin Fogg}
\IEEEauthorblockA{School of Electrical and\\Computer Engineering\\
University of New Mexico\\
bfogg@unm.edu}
\and
\IEEEauthorblockN{David Kirby}
\IEEEauthorblockA{School of Electrical and\\Computer Engineering\\
University of New Mexico\\
davidkirby@unm.edu}}

% make the title area
\maketitle

\begin{abstract}
Beginning in 1997, the field of Internet economics began to combine engineering principles with the study of economics to guide the development of the network. Over the past decades, the field has been expanded by [insert concepts from more recent literature.] These changes have shaped network economics into a more [adjective] field, as seen in the recent development of [research and technology.] However, the emergence of modern streaming services, combined with a decrease in the number of available Internet service providers, has ignited a new debate centered around the topic of "net neutrality".

The original notions of a free and open Internet [references for original values of the Internet] have been under attack from organizations that would monetize the backhaul of the Internet. This would give bandwidth and resource priority to companies that pay a premium, while startups and companies that cannot afford to pay this premium would be relegated to a slower, second class Internet.
\end{abstract}

\section{Introduction}
The application of economic principles to the engineering of the Internet received its first formal treatment by McKnight and Bailey in 1997 \cite{mcknightbailey97}. The fundamental topic explored by this combination of engineering and economics dates back even further than 1997. In 1974, Kleinrock considered the challenges of an "equitable charging and accounting scheme in such a mixed network system" \cite{kleinrock74}. Today, the proliferation of multimedia delivered via Internet has reignited these considerations under the new banner of network neutrality \cite{faulhaber11}. However, even McKnight and Bailey reference "the convergence of [...] television, telephony, and computers," along with the growing amounts of network traffic dedicated to "digital video, audio, and interactive multimedia".

\subsection{Subsection Heading Here}
Subsection text here.

\subsubsection{Subsubsection Heading Here}
Subsubsection text here.

%\begin{table}[!t]
%\renewcommand{\arraystretch}{1.3}
%\extrarowheight
%\caption{An Example of a Table}
%\label{table_example}
%\centering
%\begin{tabular}{|c||c|}
%\hline
%One & Two\\
%\hline
%Three & Four\\
%\hline
%\end{tabular}
%\end{table}

\section{Body}

\subsection{Economics of Historical Networks}
Subsection text here.

\subsubsection{ARPANET and Early Internet}
Kleinrock's 1974 paper considered the problems of "large computer communication networks," which he described as consisting of one thousand nodes or more. However, the computers discussed here are the time-sharing systems

\subsubsection{World Wide Web}
Subsubsection text here.

\subsection{Modern Network Economics}
Subsection text here.

\subsubsection{Modern Content Distribution}
As of 2015, Internet traffic to Netflix peaked at 36\% of all bandwidth consumption \cite{spangler15}
\subsubsection{ISPs and Monopoly}
Subsubsection text here.
\subsubsection{Net Neutrality}
Subsubsection text here.

\section{Conclusion}
Since 1997, Internet economics have seen [overarching description of changes], most prominently in [case], [case], and [case]. These changes [have/have not] altered the field as it was originally set forth by McKnight and Bailey, though []. In addition, the explosion of streamed media has given new consideration to a topic first examined during the Internet's infancy as ARPANET in 1974. The new banner of net neutrality, rallying against an Internet service market seen as monopolistic, has only increased concerns about the economic forces driving network traffic and policy.

% use section* for acknowledgment
\section*{Acknowledgment}
The authors would like to thank...

\begin{thebibliography}{6}

\bibliographystyle{ieee}

\bibitem{mcknightbailey97}
L.~W. McKnight and J.~P. Bailey, "An Introduction to Internet Economics," \emph{The Journal of Electronic Publishing}, vol. 1, no. 1 \& 2,\hskip 1em plus
  0.5em minus 0.4em\relax Jan. 1997

 \bibitem{kleinrock74}
 L. Kleinrock, "Research Areas in Computer Communication," \emph{Computer Communication Review}, vol. 4, no. 3, July 1974.

 \bibitem{faulhaber11}
 G.~R. Faulhaber, "Economics of net neutrality: A review," \emph{Communications \& Convergence Review}, vol. 3, no. 1, July 2011.

 \bibitem{spangler15}
  T.~Spangler, "Netflix bandwidth usage climbs to nearly 37\% of internet traffic at peak hours", \emph{Variety}, 2015. [Accessed: 28-Feb-2020].

\end{thebibliography}

\end{document}
